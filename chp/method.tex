\chapter{Implementation and Results}
As outlined in the Objectives we can divide the tasks into three main parts:
transmission, optical conversion, and reception.  In this project we looked at
using an FPGA board for the generation and reception of the PRBS data.  Hence
the overall design is of a board in a loopback configuration as shown in
Figure~\ref{fig:loopback}.

\begin{figure}[ht]
    \centering
    \includegraphics[width=0.5\linewidth]{img/loopback.png}
    \caption{Loopback Configuration}%
    \label{fig:loopback}
\end{figure}

\section{Transmission and Reception}%
\label{sec:transmission_and_reception}

\subsection{Hardware}%
\label{sub:hardware}
To generate and receive PRBS data the VCU118 board was used. The transmission
and reception of the data was handled by the onboard high-speed parallel to
serial GTY transceivers in conjunction with a Si5345 external clock. To connect
with the transceiver the HiTechGLobal FMC-MSMP module was used.
\begin{figure}[ht]
    \centering
    \includegraphics[width=0.5\linewidth]{img/board.jpg}
    \caption{VCU118 Board}%
    \label{fig:board}
\end{figure}

\begin{figure}[ht]
    \centering

    \begin{minipage}[b]{0.6\textwidth}
        \includegraphics[width=0.8\linewidth]{img/clock.jpg}
        \caption{Si5345 Clock}%
        \label{fig:clock}
    \end{minipage}
    \hfill
    \begin{minipage}[b]{0.3\textwidth}
        \includegraphics[width=0.8\linewidth]{img/fmc.png}
        \caption{FMC-MSMP module}%
        \label{fig:fmc}
    \end{minipage}

\end{figure}

\subsection{Transceiver Setup}%
\label{sub:transceiver_setup}
Most of the project took place using the transceivers in a simple RF loopback
configuration (no optical transmission).

The setup of the transceivers followed the example design outlined in the user
guide to the transceivers \cite{wizard_guide}. Places where choices were made
(as the example design is not specific to a particular board) or the
design was deviated from are outlined below. 

\subsubsection{Selection of Quads}%
\label{ssub:selection_of_quads}
The GTY transceivers in the VCU118 are grouped into four channels or quads.
Seven GTY quads on the left side of the device and six GTY quads on the right
side of the device.
There are 52 transceivers on VCU118 board, in total.
\begin{itemize}
    \item Four of the GTY transceivers are wired to Samtec Firefly Module Connector 
    \item Four of the GTY transceivers are wired to QSFP1 module connector 
    \item Four of the GTY transceivers are wired to QSFP2 module connector 
    \item Sixteen of the GTY transceivers are wired to the PCIe 16-lane edge connector
    \item Twenty-four of the GTY transceivers are wired to FMC+ HSPC connector
\end{itemize}

As we were using a FMC-HSMP module to connect to the transceivers, we were limited to the transceivers wired
to the FMC+ HSPC connector.
Furthermore only certain quads can be driven through an external clock.
As such we chose Quad 120 and Quad 122 as the quads for testing.

There are also limitations to the number of quads that can be driven with an
external clock while still meeting jitter requirements, but as we are only
driving two transmitters, it was not an issue.



\subsubsection{Pin Configuration}%
\label{ssub:pin_configuration}
It was also necessary to map the pins on the board to those on
the FMC-MSMP module. This was done as in Table~\ref{tab:pinout}.
As two transceivers were used, we mapped out the pins for the clocks, transmitters,
and receivers of both.

\noindent For a full pinout diagram refer to the user guide of the board
\cite{vcu118_guide}. 

\begin{table}[ht]
    \centering
    \hspace*{-3.5cm}\begin{tabular}{|c|c|c|c|c|c|}
        \hline
        Function & Channel & Net Name & FPGA PIN & Connected Pin & FMC+ Pin \\
        \hline
        clk\_p & GTYE4\_COMMON\_X0Y1  & FMCP\_HSPC\_GBTCLK5\_M2C\_P & AN40 & Z20 & J106 \\
        clk\_n & GTYE4\_COMMON\_X0Y1  & FMCP\_HSPC\_GBTCLK5\_M2C\_N & AN41 & Z21 & J115 \\
        tx\_p  & GTYE4\_CHANNEL\_X0Y4 & FMCP\_HSPC\_DP20\_C2M\_P    & BD42 & Z8  & J117 \\
        tx\_n  & GTYE4\_CHANNEL\_X0Y4 & FMCP\_HSPC\_DP20\_C2M\_N    & BD43 & Z9  & J116 \\
        rx\_p  & GTYE4\_CHANNEL\_X0Y4 & FMCP\_HSPC\_DP20\_M2C\_P    & BC45 & M14 & J55  \\
        rx\_n  & GTYE4\_CHANNEL\_X0Y4 & FMCP\_HSPC\_DP20\_M2C\_N    & BC46 & M15 & J54  \\
        \hline
        clk\_p & GTYE4\_COMMON\_X0Y3  & GBTCLK2\_M2C\_P             & AF38 & L12 & J21  \\
        clk\_n & GTYE4\_COMMON\_X0Y3  & GBTCLK2\_M2C\_N             & AF39 & L13 & J20  \\
        tx\_p  & GTYE4\_CHANNEL\_X0Y8 & DP0\_C2M\_P                 & AT42 & C2  & J77  \\
        tx\_n  & GTYE4\_CHANNEL\_X0Y8 & DP0\_C2M\_N                 & AT43 & C3  & J78  \\
        rx\_p  & GTYE4\_CHANNEL\_X0Y8 & DP0\_M2C\_P                 & AR45 & C6  & J100 \\
        rx\_n  & GTYE4\_CHANNEL\_X0Y8 & DP0\_M2C\_N                 & AR46 & C7  & J99  \\
        \hline
    \end{tabular}
    \caption{Transceiver Pin Out}
    \label{tab:pinout}
\end{table}


\subsubsection{Transceiver Configuration}%
\label{ssub:transceiver_configuration}
The transceivers were programmed to run at a standard rate of 10GB/s, with a
reference clock taken from the external Si5345 clock which was configured to
run at 156.25 MHz with a format of 2.5V LVDS.
The free-running clock (used to drive resets) was driven at 125 MHz, sourced
from the onboard 300 MHz system clock.  The reset functions were bound to the
physical push buttons BE23 and BB24.  Status indicators (Link Up and Link Down)
were bound to LED1 and LED0 respectively. 

At the current stage there was no encoding as we were trying to troubleshoot an
issue with the PRBS checking module. However if running the system for a long
period of time, encoding would be necessary to ensure the CDR remained locked.
This can be easily done from the transceiver wizard.
The full details of the setup can be found in
Appendix~\ref{cha:transceiver_settings}.


\cleardoublepage
\subsection{Transmitter}%
\label{sub:transmitter}
In Figure~\ref{fig:transmitter_block} we see a block diagram of the transmitter
(TX) of the transceiver. Parallel data flows into the TX interface,
is serialised, then finally flows out of the transmitter as high-speed
serial data.

\begin{figure}[ht]
    \centering
    \hspace*{-1cm}\includegraphics[width=1.2\linewidth]{img/transmitter_block.png}
    \caption{Transmitter Block Diagram~\cite{GTY_guide} }%
    \label{fig:transmitter_block}
\end{figure}

We looked to modify the functionality of a basic implementation of the
transmitter. In the basic implementation the output of a PRBS generator is fed
to the transmitter interface through a wrapper as in
Figure~\ref{fig:tx_interface}.

\begin{figure}[ht]
    \centering
    \includegraphics[width=0.6\linewidth]{img/tx_interface.png}
    \caption{Generator to TX Interface}%
    \label{fig:tx_interface}
\end{figure}

\noindent There were two variations of the PRBS generation module that were developed.

\subsubsection{Burst Mode over Single Channel}%
\label{ssub:burst_mode_over_single_channel}
Here we modified the PRBS generation module by setting it to output zeros if the
enable command was not asserted.  In combination we added a register
to the wrapper, which on overflow toggles the enable flag. This has the
effect of causing the PRBS module to output a sequence interspersed with zeros.
The code can be found in Section~\ref{sec:single_channel_burst_mode}.

A testbench where the unmodified PRBS generator and burst PRBS generator
are compared is shown in in Figure~\ref{fig:burst_mode_start}.  We see the
burst generator generating the expected result: a PRBS sequence interspersed
with zeros. We note also that the full sequence is transmitted in burst mode (no
data is lost).

\begin{figure}[ht]
    \centering
    \hspace*{-3cm}\includegraphics[width=1.4\linewidth]{img/burst_mode_1.png}
    \caption{Single Burst Mode Behaviour}%
    \label{fig:burst_mode_start}
\end{figure}

Comparing Figure~\ref{fig:normal_mode_recur} and
Figure~\ref{fig:burst_mode_recur} we see that the recurrence time (in this
testbench we use PRBS7, as it is a short sequence) for the normal sequence is
about half that of the burst mode sequence. This is expected as the burst mode
sequence is operating on a duty cycle of 50\% and hence should take double the
time to recur.

\begin{figure}[ht]
    \centering
    \hspace*{-3cm}\includegraphics[width=1.4\linewidth]{img/burst_mode_2.png}
    \caption{Normal Mode Recur Time}%
    \label{fig:normal_mode_recur}
\end{figure}

\begin{figure}[ht]
    \centering
    \hspace*{-3cm}\includegraphics[width=1.4\linewidth]{img/burst_mode_3.png}
    \caption{Burst Mode Recur Time}%
    \label{fig:burst_mode_recur}
\end{figure}


\subsubsection{Switching Between Two Channels}%
\label{ssub:switching_between_two_channels}
Here the PRBS wrapper was changed to feed the two different outputs. The PRBS
generator was unchanged. We added a register which on overflow alternated the
output of the generator module. The code can be found in Section~\ref{sec:two_channel_burst_mode} 
This had the overall effect of having the whole sequence be sent over two
different channels as shown in Figure~\ref{fig:two_channel_tx}.


\begin{figure}[ht]
    \centering
    \hspace*{-3cm}\includegraphics[width=1.4\linewidth]{img/two_channel.png}
    \caption{Two Channel Burst Mode}%
    \label{fig:two_channel_tx}
\end{figure}

\cleardoublepage

\subsection{Receiver}%
\label{sub:receiver}
In Figure~\ref{fig:receiver_block} we see a block diagram of the receiver
(RX) of the transceiver. Serial data flows into the reciever, is deserialised,
and is finally passed to the RX Interface where it can then be passed to the
rest of the board. 
\begin{figure}[ht]
    \centering
    \hspace*{-1cm}\includegraphics[width=1.2\linewidth]{img/receiver_block.png}
    \caption{Receiver Block Diagram \cite{GTY_guide}}%
    \label{fig:receiver_block}
\end{figure}

In the basic implementation the RX interface passes data to the PRBS checker
through a wrapper. We looked to modify the wrapper so that burst mode checking
would work correctly.
\begin{figure}[ht]
    \centering
    \includegraphics[width=0.6\linewidth]{img/rx_interface.png}
    \caption{RX Interface to Checker}%
    \label{fig:rx_interface}
\end{figure}
\cleardoublepage
\subsubsection{Burst Mode Checking}%
\label{ssub:burst_mode_checking}
In burst mode checking there are periods when the incoming bitstream is all
zeros. The PRBS checker module takes the incoming data as a seed to calculate
the next expected sequence. If zeros are provided then this interferes with the
module (as the next expected word will be calculated based on zeros). To
compensate for this we loaded incoming data to a register and checked if it was
zeros. If not, it was passed to the PRBS module.
The code can be found in Section~\ref{sec:burst_mode_checking}.  

In simulation when the PRBS generation module was connected directly to the
PRBS checker this worked correctly as shown in Figure~\ref{fig:burst_check_1}
with no errors (the sequence was repeated nearly 4,000 times). 
\begin{figure}[ht]
    \centering
    \hspace*{-3cm}\includegraphics[width=1.5\linewidth]{img/burst_check.png}
    \caption{Direct Connection Checking}%
    \label{fig:burst_check_1}
\end{figure}

However, when passed through the transceiver, the PRBS checker found errors
consistently, as shown in Figure~\ref{fig:burst_check_2}.  These errors appear
to originate in situations where zeros are sent to the TX interface, but some
bits are received at the RX interface.  We suspect that this may be due to some
encoding by the transceiver, but were unable to pinpoint the exact cause before the
end of the project. 
There are other workarounds that may have been possible, such as
loading the entire sequence into memory and processing it there to remove the
zeros, or by sending shorter sequences, so that each burst would contain a full
sequence, but we were not able to explore this fully.

\begin{figure}[ht]
    \centering
    \hspace*{-3cm}\includegraphics[width=1.5\linewidth]{img/burst_check_2.png}
    \caption{Transceiver Checking}%
    \label{fig:burst_check_2}
\end{figure}

\subsubsection{Two Channel Checking}%
\label{ssub:two_channel_checking}
In the case where two transmitters were muxed together and were sent to a
single receiver, it should not have been necessary to change the behaviour of
the PRBS checking module.
We were unable to check this due to the closing of the lab.

\subsubsection{Disabling The CDR}%
\label{ssub:disabling_the_cdr}
The receiver CDR block detailed block diagram is shown in
Figure~\ref{fig:cdr_detail}. We see it operates as a control system similar to
what is described in Section~\ref{ssub:clock_and_data_recovery}:
incoming data is sampled by edge and data samplers to get phase information,
which is passed to a state machine which makes decisions on how to adjust
the phase of the local clock (provided by the PLL) using phase interpolators.

\begin{figure}[ht]
    \centering
    \hspace*{-1cm}\includegraphics[width=1.2\linewidth]{img/cdr_detail}
    \caption{CDR Block Diagram}%
    \label{fig:cdr_detail}
\end{figure}

The final goal of the project would have been to run the receiver
source-synchronously.  The transceiver did not allow much flexibility here, and
were limited to the configuration options that were offered by the Xilinx IP.
We were not able to bypass the CDR entirely. We attemped to mimic this a much
as possible by locking the CDR to reference clock and using the same clock to
drive the receiver and transmitter, ideally then taking readings of phase, and
making the necessary compensations.
 
We were able to disable the CDR by asserting the CDRHOLD ports. This was done
by hardcoding the CDRHOLD value. We further extended this by adding CDRHOLD in
conjunction with the QPLLLOCK and RXCDRLOCK ports to the Xilinx Virtual
Input/Output (VIO) which allowed us to toggle the CDR.  The link became
unstable after the port was asserted.  

We were then unable to read the phase. The phase caching paper by Kari, et al.
\cite{kari_phase} was done on Xilinx Ultra-Scale board and they were able to
read and modify the phase by reading/writing values to the appropriate register.
However the details of that implementation were specific to the Ultra-Scale
boards. The board used in the project is an Ultra-Scale+ board and there is no
publicly available information on how to modify/read the phase. 
As such we were unable to explore this further.

\cleardoublepage
\section{Optical Transmission}%
\label{optical_transmission}
We were not able to test this hardware in conjuction with the burst mode
transmission, as the labs closed due to external factors. However the boards
were used in other projects that took place in the research group, including a
recent paper where machine learning was used to optimise the switching times of
the SOA \cite{gerard2020swift}.

\subsection{SOA Board}%
\label{sub:soa_board}
The design consisted of a commercial Inphenix IPSAD1502 using input from a 
matched RF line to modulate an incoming laser. The SOA was also temperature
controlled at 25 degrees Celsius. 
The driving current was 200mA. As the input resistors are $50\Omega$, the power
dissipation is 2W, and the resistors must be rated as such.

\noindent
The circuit diagram can be seen in Figure~\ref{fig:soa_circuit} (developed by
Dennis Goh).

\begin{figure}[ht]
    \centering
    \hspace*{-1cm}\includegraphics[width=1.3\linewidth]{img/soa_circuit.png}
    \caption{SOA circuit}%
    \label{fig:soa_circuit}
\end{figure}

\noindent
4 PCBs were prepared and made. The PCB design can be found in Appendix~\ref{cha:soa_pcb}.
The Temprature Controller was connected to the IC uing a 9-pin D-type
connector.

\cleardoublepage 
\subsection{Heatsink and Mount}%
\label{sub:heatsink_and_mount}
It was also necessary to attach a aluminium substrate underneath the PCB board
for heat dissipation. An Aluminium substrate was machined to match the PCB (as shown in
Appendix~\ref{cha:substrate}), then the SOA was placed in contact with the
substrate.  We then stacked the four boards above each other as in
Figure~\ref{fig:full_mount}.

\begin{figure}[ht]
    \centering
    \includegraphics[width=0.8\linewidth]{img/full_mount.jpeg}
    \caption{Stacked Boards}%
    \label{fig:full_mount}
\end{figure}
