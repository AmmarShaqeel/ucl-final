\chapter{Theoretical Basis}

\section{Background Theory}%
\label{sec:background_theory}

\subsubsection{Clock and Data Recovery}%
\label{ssub:clock_and_data_recovery}
Commonly a serial data stream is sent over a channel without a clock signal.
Clock recovery is the process of extracting timing information from a serial
data stream. This timing data is then used to decode the received data stream.
This process is known as Clock and Data Recovery (CDR).

In this application a bang-bang CDR circuit is used. 
In this case transitions in a received data signal are counted as "early" or
"late" as compared with a local clock. The clock can then be adjusted based on
the local transitions\cite{alexander1975clock}.

\subsubsection{Source Synchronous System}%
\label{ssub:source_synchronous_system}
In a source synchronous system a clock signal is provided alongside the data
signal. This allows us.

\section{Literature Review}%
\label{sec:literature_review}

\noindent \cite{kari_phase} outlines how CDR circuits are a limiting factor in
optical switching and proposes a method of overcoming this (phase caching).
Through phase caching they were able to demonstrate sub nanosecond locking
times, improving the locking time 12x.

\noindent In \cite{serrano2013white}, \cite{moreira2010digital}, and
\cite{moreira2009white}, the white rabbit project is discussed. A white rabbit
system provides sub-nanosecond synchronisation accuracy. To achieve this,
accurate measurements of the link delay between the nodes of the network must
be calculated.  While instructive, the method may not be directly applicable to
the project, as in a White Rabbit system, all the nodes are locked to the same
frequency. Hence the link delay can be calculated by having a node receive a
clock signal from another node, then return the same signal. The link delay can
then be calculated by comparing the phase offset of the two signals.

\noindent \cite{williams2016source} described an optical source synchronous
system. It describes how choosing the correct wavelength for the clock can
minimise the modal cross-talk. Furthermore, in conjunction with
\cite{ragab2011receiver} it was quite instructive in describing how source
synchronous systems are able to track correlated jitter between clock and data
channels, and how system performance can be degraded by channel slew between
clock and data channels.\\ \cite{williams2019reconfiguration} further explored
reducing the modal crosstalk by proposing an architecture with re-configurable
clock and data paths, thus allowing the user to chose the optimal lane for the
sensitive clock for each photonic interconnect. This may not be needed however,
as each transmitter should have a fixed data characteristic.

\noindent \cite{chen2017optimization} and \cite{fixed_latency} describe fixed
latency links. There is a possibility that we will be unable to completely
bypass the CDR. In this case if we can organise the system to have fixed
latency, we could force the CDR to the appropriate fixed phase. Thus the
circuit would then be able to have a stable source-synchronous link.

\noindent \cite{dru_guide}, \cite{nidru} describe an intellectual property (IP)
that allows the high speed serial transceivers to be used at much lower data
rates. This was initially of interest because it would have been easier to
demonstrate a working system with lower data rates. However as this is an extra
IP used in conjunction with the transceivers it may not be useful for the
project. 

\noindent \cite{mendes_transceiver} this presentation describes a system where
the phase of a transceiver on Xilinx board is kept stable over resets. While
this was done primarily on the transmitter side (we are interested in the
receiver side) this still shows that fixing the phase of the transceiver is
possible in some cases.

