\documentclass[oneside]{discothesis}
\usepackage[outputdir=build]{minted}
\setminted[python]{frame=lines, fontsize=\footnotesize, linenos}    

%%%%%%%%%%%%%%%%%%%%%%%%%%%%%%%%%%%%%%%%%%%%%%%%%%%%%%%%%%%%%%%%%%%%%%%%%%%%%%%%%%%%%%%%%%%%%%%%%
% DOCUMENT METADATA

\thesistype{A MEng Project Final Report} % Master's Thesis, Bachelor's Thesis, Semester Thesis, Group Project
\title{Clock and Data Recovery over Optical Links and Networks}

\author{Ammar Bin Shaqeel Ahmed}
\email{ammar.ahmed.16@ucl.ac.uk \\ 16080322}

\institute{University College London}

% Optionally, you can put in your own logo here
%logo{\includegraphics[width=0.2\columnwidth]{figures/disco_logo_faded}}

\supervisors{Dr. Georgios Zervas}
\assessor{Dr. Domaniç Lavery}

% Optionally, keywords and categories of the work can be shown (on the Abstract page)
%\keywords{Keywords go here.}
%\categories{ACM categories go here.}

\date{May, 2020}

%%%%%%%%%%%%%%%%%%%%%%%%%%%%%%%%%%%%%%%%%%%%%%%%%%%%%%%%%%%%%%%%%%%%%%%%%%%%%%%%%%%%%%%%%%%%%%%%%

\begin{document}

\frontmatter % do not remove this line
\maketitle
\cleardoublepage

\begin{acknowledgements}
 Lorem ipsum dolor sit amet, consectetur adipiscing elit. Nunc blandit tellus
 id lectus egestas, sit amet finibus diam eleifend. Nulla nulla felis,
 hendrerit vel mi ac, tempus varius libero. Proin lobortis sapien eget
 malesuada aliquet. Nulla ac eleifend velit. Suspendisse feugiat magna
 convallis, consectetur leo a, ultrices ex. Aliquam suscipit mi eu tempor
 porttitor. In facilisis eget massa ac iaculis. Sed dolor lacus, scelerisque in
 diam vitae, finibus aliquam lorem. Donec eu leo eu lorem sodales varius. Donec
 dictum mauris sed ornare porttitor. Nunc imperdiet eu elit non hendrerit. Sed
 velit lectus, lacinia sed dolor vitae, faucibus eleifend felis. Aliquam ac ex
 enim. Aliquam erat volutpat. Class aptent taciti sociosqu ad litora torquent
 per conubia nostra, per inceptos himenaeos. Pellentesque pulvinar sollicitudin
 mattis.
\end{acknowledgements}

\begin{abstract}
 Lorem ipsum dolor sit amet, consectetur adipiscing elit. Nunc blandit tellus
 id lectus egestas, sit amet finibus diam eleifend. Nulla nulla felis,
 hendrerit vel mi ac, tempus varius libero. Proin lobortis sapien eget
 malesuada aliquet. Nulla ac eleifend velit. Suspendisse feugiat magna
 convallis, consectetur leo a, ultrices ex. Aliquam suscipit mi eu tempor
 porttitor. In facilisis eget massa ac iaculis. Sed dolor lacus, scelerisque in
 diam vitae, finibus aliquam lorem. Donec eu leo eu lorem sodales varius. Donec
 dictum mauris sed ornare porttitor. Nunc imperdiet eu elit non hendrerit. Sed
 velit lectus, lacinia sed dolor vitae, faucibus eleifend felis. Aliquam ac ex
 enim. Aliquam erat volutpat. Class aptent taciti sociosqu ad litora torquent
 per conubia nostra, per inceptos himenaeos. Pellentesque pulvinar sollicitudin
 mattis.

Duis vitae malesuada tellus, malesuada tincidunt sem. Nam dapibus mattis lorem,
sagittis egestas risus ullamcorper a. Etiam finibus, libero non hendrerit
scelerisque, nisl nisi blandit lacus, id laoreet tortor turpis sit amet sapien.
Nullam non lacus ut tortor consequat auctor. Vivamus congue, massa in suscipit
imperdiet, dui neque ultricies purus, ut fermentum sapien ante vel urna. Sed at
blandit ligula. Sed non enim ante. Integer volutpat vestibulum leo, eu
malesuada diam lacinia vel. Nulla ultrices leo eu cursus feugiat. Morbi semper
nisl ex. Etiam volutpat consectetur sodales. Vestibulum facilisis dapibus enim,
ac rutrum massa blandit vehicula. Aenean nec lectus varius, congue risus at,
volutpat purus. Duis id rhoncus felis. 
\end{abstract}

\tableofcontents

\mainmatter % do not remove this line

% Start writing here
\chapter{Introduction}
Bandwidth demands in data centers have been doubling every 12-15 months. For
data center providers to keep pace with the increased demand (at the same price
point) network switches have had to double their capacity while staying at
roughly the same cost \cite{singh2016jupiter}. However this trend seems to be
coming to an end for two reasons. The first is a predicted increase in the rate
of growth of demand, due to trends like hardware accelerated programming and
dis-aggregated workloads. The second is because electrical switches are
predicted to reach a limit due to the physical limits on pin density
\cite{ballani2018bridging}.

For these reasons optical switching is being explored, as it has the potential
to overcome many of these problems. Optical switches do not require
opto-electrical (OEO) conversion, and hence the number of expensive and power
hungry transceivers required is reduced. Furthermore, as buffering is not
needed, the latency of the optical switches is much lower. Lastly, they do not
use electronics for switching, thus bypassing the aforementioned physical limit
\cite{ballani2018bridging}. 

In data centers much of the traffic that is transmitted between servers is in
the form of small data packets, with 97.8\% of packets being 576 bytes or
less~\cite{data_size}. With 100 Gb/s ports this means that switching should
take place on the order of hundreds of nanoseconds. 

When data is transmitted without a clock signal, the clock has to be
regenerated at the receiver before the data can be decoded - this is known as
clock and data recovery (CDR). The time taken for the local clock to "lock" to
the data stream, adds latency. 
In optical switches physical links are created between each
transceiver-receiver pair. Hence each time the switch is reconfigured, the CDR
must re-lock to the new link.  This means that the network throughput is
limited by the sum of the optical switching time and the CDR locking time -
which can be hundreds of nanoseconds in the worst case and tens of
nanoseconds in the best case~\cite{Chen:18}. Assuming an optical switching time
of 1 nanosecond, it is evident that the CDR locking time acts as bottleneck
that can drastically reduce the throughput~\cite{kari_phase}.

In a source synchronous system the clock is transmitted alongside the data,
removing the CDR locking time. This would remove the bottleneck, theoretically
increasing the throughput.


\chapter{Theoretical Basis}

\section{Background Theory}%
\label{sec:background_theory}
Here we go deeper into the theory of certain elements of the system.

\subsubsection{Bang-Bang CDR}%
\label{ssub:clock_and_data_recovery}
Commonly a serial data stream is sent over a channel without a clock signal.
Clock and Data Recovery (CDR) is the process of extracting timing information
from a serial data stream, then using it to decode the received data stream.  A
CDR circuit has two primary functions. The first is to extract a clock based on
the input data, and the second is to resample the data. 

To extract the clock from the data, a local clock is generated, then is adjusted
as "early" or "late" when compared with the incoming data
signal \cite{sun1989analog}.  We can think of this as a control system, as
shown in Figure~\ref{fig:cdr_basic}.

\begin{figure}[ht]
    \centering
    \includegraphics[width=1\linewidth]{img/cdr_basic.png}
    \caption{Basic CDR design}%
    \label{fig:cdr_basic}
\end{figure}

Phase detectors can be divided into two types, linear (where the output has a
linear relationship to the input) and binary or bang-bang phase detectors (where the output
is either positive or negative). Binary phase detectors are more commonly used
in digital CDR circuits~\cite{ZHANG2015163}. An example of one is the
Alexander detector \cite{alexander1975clock} which gives out a high D0+ and a low D0- if the
clock lags and vice-versa if the clock leads, as shown in
Figure~\ref{fig:bang_bang}.

\begin{figure}[ht]
    \centering
    \includegraphics[width=0.8\linewidth]{img/bang_bang.png}
    \caption{Alexander Phase Detector}%
    \label{fig:bang_bang}
\end{figure}

\subsubsection{Pseudorandom Binary Sequence}%
\label{ssub:prbs_generation}
A pseudorandom binary sequence (PRBS) is a sequence of bits that appears to be
random. However as it is generated using a deterministic algorithim, it can be
replicated if the inital conditions are the same.

A common practical implementation of PRBS generation uses linear-feedback shift registers.  As an
example, a PRBS-4 sequence could be generated by using a 4 bit register. We
seed the register with a non-zero number, then tap two bits of the register as
an input. We then shift the contents of the register, taking the last bit as an
output and the new bit as an input, as illustrated in
Figure~\ref{fig:img/shift_reg}.

\begin{figure}[ht]
    \centering
    \includegraphics[width=0.4\linewidth]{img/shift_reg.png}
    \caption{Shift Register Implementation}%
    \label{fig:img/shift_reg}
\end{figure}

The full operation can be seen in Table~\ref{tab:shift_reg}. As 0000 cannot
appear (the value of the register would never change) we see that for a register of size N, the
bitsequence is $2^N - 1$ bits long. 
\begin{table}[ht]
    \centering
    \begin{tabular}{|c|c|c c c c|c|}
    \hline
    Cycle & Input & \multicolumn{4}{|c|}{Shift Register} & Output \\
    \hline
     0  & 1 & 1 & 0 & 0 & 1 & 1 \\
     1  & 0 & 1 & 1 & 0 & 0 & 0 \\
     2  & 1 & 0 & 1 & 1 & 0 & 0 \\
     3  & 0 & 1 & 0 & 1 & 1 & 1 \\
     4  & 1 & 0 & 1 & 0 & 1 & 1 \\
     5  & 1 & 1 & 0 & 1 & 0 & 0 \\
     6  & 1 & 1 & 1 & 0 & 1 & 1 \\
     7  & 1 & 1 & 1 & 1 & 0 & 0 \\
     8  & 0 & 1 & 1 & 1 & 1 & 1 \\
     9  & 0 & 0 & 1 & 1 & 1 & 1 \\
     10 & 0 & 0 & 0 & 1 & 1 & 1 \\
     11 & 1 & 0 & 0 & 0 & 1 & 1 \\
     12 & 0 & 1 & 0 & 0 & 0 & 0 \\
     13 & 0 & 0 & 1 & 0 & 0 & 0 \\
     14 &   & 0 & 0 & 1 & 0 & 0 \\
    \hline
    \end{tabular}
    \caption{Shift Register Operation}
    \label{tab:shift_reg}
\end{table}

\subsubsection{Source Synchronous System}%
\label{ssub:source_synchronous_system}
In a source synchronous system a clock signal is provided alongside the data
signal, as shown in Figure~\ref{fig:source_sync}. This has the advantage of not
needing a CDR circuit. Furthermore as both the clock and the data come from the
same device any jitter will be similar across both signals and can likely be
ignored \cite{ragab2011receiver}.   A downside is that there will be crossing
of clock domains at the reciever as the transmitted clock will not be
synchronous with the clock domain of the receiving device.
\begin{figure}[ht]
    \centering
    \includegraphics[width=0.5\linewidth]{img/source_sync.png}
    \caption{Source Synchronous System}%
    \label{fig:source_sync}
\end{figure}



\subsubsection{Semiconductor Optical Amplifier}%
\label{ssub:semiconductor_optical_amplifier}
Optical amplifiers are devices that can amplify an optical signal without
needing to convert it to an electrical one. 
A silicon optical amplifier (SOA) is one that uses a semiconductor as the gain
medium, as light passes through this gain medium it is amplified.
SOAs are electrically pumped (do not require the use of another laser) and are of
small size.
\begin{figure}[ht]
    \centering
    \includegraphics[width=0.4\linewidth]{img/soa.png}
    \caption{Basic SOA Structure}%
    \label{fig:soa}
\end{figure}



\section{Literature Review}%
\label{sec:literature_review}

\noindent \cite{kari_phase} outlines how CDR circuits are a limiting factor in
optical switching and proposes a method of phase caching to overcome this.
The data is transferred over the high-speed Xilinx transceivers, and uses a
bang-bang CDR. The phase measurments that are being compared are that of the
recieved data to the local clock.
The PRBS data is pregenerated (written to memory) and is sent in short bursts
with a known sequence at the end. When the data arrives it is then written to
memory and then processed.  The phase caching improved locking time on
switching by 12 times.

\noindent In \cite{serrano2013white}, \cite{moreira2010digital}, and
\cite{moreira2009white}, the white rabbit project is discussed. A white rabbit
system provides sub-nanosecond synchronisation accuracy. To achieve this,
accurate measurements of the link delay between the nodes of the network must
be calculated.  While instructive, the method is not directly applicable to the
project, as in a White Rabbit system, all the nodes are locked to the same
frequency. Hence the link delay can be calculated by having a node receive a
clock signal from another node, then return the same signal. The link delay can
then be calculated by comparing the phase offset of the two signals.

\noindent \cite{williams2016source} described an optical source synchronous
system. It describes how choosing the correct wavelength for the clock can
minimise the modal cross-talk. Furthermore, in conjunction with
\cite{ragab2011receiver} it describes how source synchronous systems are able
to track correlated jitter between clock and data channels, and how system
performance can be degraded by channel slew between clock and data channels.\\
\cite{williams2019reconfiguration} further explored reducing the modal
crosstalk by proposing an architecture with re-configurable clock and data
paths, thus allowing the user to chose the optimal lane for the sensitive clock
for each photonic interconnect. This may not be needed however, as each
transmitter should have a fixed data characteristic.

\noindent \cite{chen2017optimization} and \cite{fixed_latency} describe fixed
latency links. In the event we were unable to bypass the CDR, it may be
possible to organise the system to have a fixed latency, then force the CDR to
the appropriate fixed phase. Thus the circuit could thus have a much reduced
CDR lock time.

\noindent \cite{dru_guide}, \cite{nidru} describe an Xilinx intellectual
property that allows the high speed serial transceivers to be used at much
lower data rates. This was initially of interest because it would have been
easier to demonstrate a working system with lower data rates. However as this
is an extra IP used in conjunction with the transceivers it did not turn out to
be useful for the project. 

\noindent \cite{mendes_transceiver} this presentation describes a system where
the phase of a transceiver on Xilinx board is kept stable over resets. While
this was done on the transmitter side it shows that fixing the phase of the
transceiver is possible.


\chapter{System Overview and Objectives}
To demonstrate the efficacy of a source synchronous system we use a single
receiver that receives data from two different channels.  The two channels
would transmit a pseudorandom binary sequence (PRBS), from a single source. At
set intervals the source alternates the channel over which it transmits.  Hence
the overall effect is that the channels would optically transmit bursts of data
at non-overlapping intervals. If the receiver is successfully able to receive
the full sequence, then the source synchronous system would be working
correctly.  An overview of the system is shown in
Figure~\ref{fig:overview}. 

\begin{figure}[h]
    \centering
    \includegraphics[width=1\linewidth]{img/overview.png}
    \caption{Overview of System}%
    \label{fig:overview}
\end{figure}

The overall objective is to demonstrate successful burst source synchronous
communication.  If successful, we would then be able to compare the throughput
of the system with a similar system that utilised a CDR circuit.

To accomplish this the following components are needed:
\begin{itemize}
    \item A burst mode PRBS generator
    \item An optical switch 
    \item Source synchronous PRBS checker 
\end{itemize}


\chapter{Implementation and Results}
In this section we cover the implementation of the project and the results.
As outlined in the Objectives section we can divide the tasks into three main
parts: generation, transmission, and reception. 
In this project we looked at using a FPGA board for the generation and
reception of the PRBS data.  Hence the overall design is of a board in a
loopback configuration as shown in Figure~\ref{fig:loopback}.

\begin{figure}[ht]
    \centering
    \includegraphics[width=0.6\linewidth]{img/loopback.png}
    \caption{Loopback Configuration}%
    \label{fig:loopback}
\end{figure}

\section{Generation and Reception}%
\label{sec:generation_and_reception}


\subsection{Hardware}%
\label{sub:hardware}
To generate and receive PRBS data the VCU118 board was used. The transmission
and reception of the data was handled by the onboard high-speed parallel to serial
GTY transceivers in conjuction with a Si5345 external clock (as the board is
not able to generate an internal clock to the needed precision).  The
full details of the setup can be found in the appendix.

\begin{figure}[ht]
    \centering
    \includegraphics[width=0.4\linewidth]{img/board.jpg}
    \caption{VCU118 Board}%
    \label{fig:board}
\end{figure}

\begin{figure}[ht]
    \centering
    \includegraphics[width=0.4\linewidth]{img/clock.jpg}
    \caption{Si5345 Clock}%
    \label{fig:clock}
\end{figure}

\subsection{PRBS Generation}%
\label{sub:prbs_generation}

We looked to modify the functionality of a basic implementation of the
transceiver. In the basic implementation a PRBS generator is is fed to the
transceiver channel, through a wrapper. 
\TODO{image of prbsgen wrapped and passed to transceiver}

The PRBS module was unchanged from the default with the exception of reducing
the length of the PRBS sequence from PRBS31 (2.1 billion bits) to PRBS7 (511
bits) for ease of checking.
There were two variations of the PRBS generation module that were developed.

\subsubsection{Burst Mode over Single Channel}%
\label{ssub:burst_mode_over_single_channel}
Here we modified the PRBS generation module further and set it to output zeros
if the enable command was not asserted.  In combination we placed a 2 bit register inside
the wrapper, which on overflow toggles the enable flag. This has the effect of
causing the PRBS module to output a sequence interspersed with zeros.

\TODO{burst mode modification}

\subsubsection{Switching Between Two Channels}%
\label{ssub:switching_between_two_channels}
The main modification was to change the PRBS wrapper to feed the two different
outputs. The PRBS generator module was unchanged. Using a 2 bit register which
on oveflow alternated between which of the outputs the PRBS data was sent to, with
the other output being sent zeros. This had the overall effect of having the
whole sequence be sent over two different channels.

\TODO{two channel switch (showing module with two outputs)}


\subsection{PRBS Checking}%
\label{sub:prbs_checking}
In normal operation the transceiver would parallise the serial data, and then
pass the data to the PRBS checking module.  
\TODO{image of transceiver -> prbs module}

\subsubsection{Two Channel Checking}%
\label{ssub:two_channel_checking}
In the case where two transmitters were muxed together and were sent to a
single receiver, it should not have been necessary to change the behaviour of
the PRBS checking module.
However we were unable to check this as the lab was closed.

\subsubsection{Burst Mode Checking}%
\label{ssub:burst_mode_checking}
For burst mode checking there are some issues as there are periods when the
incoming bitstream is all zeros. The PRBS checker module takes the incoming
data as a seed to calculate the next expected sequence. If zeros are provided
then this interferes with the module (as the next expected word will be calculated
based on zeros). To compensate for this we added a register to the wrapper that
would not pass zeros to the checker module. In a basic simulation of PRBS generation
to PRBS checker this worked correctly, but when passed to through the
transceiver, the checker module would throw errors.  We were not able to determine why.

\subsubsection{Source Synchronous Reception}%
\label{ssub:source_synchronous_reception}
The final step would have been to run the reciever source synchronously. The
transceiver did not allow much flexibilty here. We attemped to do this was by
disabling the CDR and using the same clock to drive the reciever and
transmitter. However the link here was not stable, and we were unable to get
phase readouts that may have allowed us to modify the phase of the incoming
clock. Overall this was unsuccessful.


\section{Optical Transmission}%
\label{optical_transmission}
This part of the project was not completed as the labs were closed before we
were able to test it. However some hardware was prepared, and is described in
the following sections.
\subsection{SOA Board}%
\label{sub:soa_board}

\subsection{Heatsink and Mount}%
\label{sub:heatsink_and_mount}






\chapter{Conclusion}
In this project, we aimed to achieve a fully working burst-mode
source-synchronous optical transmission. 

We developed a wrapper for a PRBS generator that enabled burst mode
transmission and were able to successfully demonstrate this in simulation.
We also developed hardware mounts for the SOAs. We were not able to test optical
transmission as the labs were closed. 

On the receiver side we developed a burst-mode receiver which worked in simulation, but
had issues when tested in the full system, that we did not have time to troubleshoot. 
Given more time it is possible that these issues could have been resolved.  

With regards to running the receiver source-synchronously we were not able to
bypass the CDR.  We attempted a workaround, but were hindered due to
a lack of publically available information about certain aspects of the
transceiver. It is possible that if a different board was used, or a different
type of transmitter, that this could have been accomplished. 

Overall we were able to progress towards a workable prototype, even though we
were not fully able to accomplish the stated goal. 


% This displays the bibliography for all cited external documents. All references have to be defined in the file references.bib and can then be cited from within this document.
\bibliographystyle{IEEEtran}
\bibliography{ref}

% This creates an appendix chapter, comment if not needed.
\appendix
\chapter{Transceiver Settings}%
\label{cha:transceiver_settings}



The configuration of the Transceiver Wizard can be found here.

\begin{figure}[ht]
    \centering
    \hspace*{-2cm}\includegraphics[width=1.3\linewidth]{img/transceiver1.png}
    \caption{Transceiver Wizard Settings}%
    \label{fig:transceiver1}
\end{figure}

\cleardoublepage

\begin{figure}[t]
    \centering
    \hspace*{-2cm}\includegraphics[width=1.3\linewidth]{img/transceiver2.png}
    \caption{Transceiver Wizard Settings 2}%
    \label{fig:transceiver2}
\end{figure}

\cleardoublepage

\begin{listing}
\begin{minted}[linenos,numbersep=5pt,frame=lines,framesep=2mm]{c}

set_property PACKAGE_PIN AN41 [get_ports mgtrefclk0_x0y1_n] 
set_property PACKAGE_PIN AN40 [get_ports mgtrefclk0_x0y1_p]

set_property IOSTANDARD DIFF_SSTL12 [get_ports hb_gtwiz_reset_clk_freerun_in_p]

set_property PACKAGE_PIN AY23 [get_ports hb_gtwiz_reset_clk_freerun_in_n]
set_property PACKAGE_PIN AY24 [get_ports hb_gtwiz_reset_clk_freerun_in_p]
set_property IOSTANDARD DIFF_SSTL12 [get_ports hb_gtwiz_reset_clk_freerun_in_n]

set_property package_pin BE23 [get_ports hb_gtwiz_reset_all_in] 
set_property IOSTANDARD LVCMOS18 [get_ports hb_gtwiz_reset_all_in]

set_property PACKAGE_PIN BB24 [get_ports link_down_latched_reset_in]
set_property IOSTANDARD LVCMOS18 [get_ports link_down_latched_reset_in]


# LED1 (working correctly) 
set_property PACKAGE_PIN AV34 [get_ports link_status_out] 
set_property IOSTANDARD LVCMOS12 [get_ports link_status_out]

# LED0 (not working) 
set_property PACKAGE_PIN AT32 [get_ports link_down_latched_out] 
set_property IOSTANDARD LVCMOS12 [get_ports link_down_latched_out]

# Clock constraints for clocks provided as inputs to the core 
------------------------------------------------------------------------------
create_clock -period 8.000 -name clk_freerun [get_ports hb_gtwiz_reset_clk_freerun_in_p] 
create_clock -period 6.400 -name clk_mgtrefclk0_x0y1_p [get_ports mgtrefclk0_x0y1_p]

# False path constraints #
-------------------------------------------------------------------------------
set_false_path -to [get_cells -hierarchical -filter {NAME =~
*bit_synchronizer*inst/i_in_meta_reg}] 
set_false_path -to [get_cells -hierarchical -filter {NAME =~
*reset_synchronizer*inst/rst_in_*_reg}]
set_false_path -to [get_pins -filter REF_PIN_NAME=~*D -of_objects [get_cells
-hierarchical -filter {NAME =~ *reset_synchronizer*inst/rst_in_meta*}]]
set_false_path -to [get_pins -filter REF_PIN_NAME=~*PRE -of_objects [get_cells
-hierarchical -filter {NAME =~ *reset_synchronizer*inst/rst_in_meta*}]]
set_false_path -to [get_pins -filter REF_PIN_NAME=~*PRE -of_objects [get_cells
-hierarchical -filter {NAME =~ *reset_synchronizer*inst/rst_in_sync1*}]]
set_false_path -to [get_pins -filter REF_PIN_NAME=~*PRE -of_objects [get_cells
-hierarchical -filter {NAME =~ *reset_synchronizer*inst/rst_in_sync2*}]]
set_false_path -to [get_pins -filter REF_PIN_NAME=~*PRE -of_objects [get_cells
-hierarchical -filter {NAME =~ *reset_synchronizer*inst/rst_in_sync3*}]]
set_false_path -to [get_pins -filter REF_PIN_NAME=~*PRE -of_objects [get_cells
-hierarchical -filter {NAME =~ *reset_synchronizer*inst/rst_in_out*}]]


set_property C_CLK_INPUT_FREQ_HZ 300000000 [get_debug_cores dbg_hub]
set_property C_ENABLE_CLK_DIVIDER false [get_debug_cores dbg_hub] set_property
C_USER_SCAN_CHAIN 1 [get_debug_cores dbg_hub] connect_debug_port dbg_hub/clk
[get_nets hb_gtwiz_reset_clk_freerun_buf_int]
\end{minted}
\caption{Constraints}
\label{lst:constraints}
\end{listing}



\end{document}
